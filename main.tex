% Pacotes
\documentclass[12pt]{article}
\usepackage{adjustbox}
\usepackage[utf8]{inputenc}
\usepackage{amsmath}
\usepackage{sbc-template}
\usepackage{amsfonts}
\usepackage{fancyvrb}
\usepackage{amsmath}
\usepackage{graphicx,url}


\sloppy

\title{Cálculo da função seno através da série de Taylor\\com dez dígitos de precisão}

%TODO: Inserir nome
\author{Ricardo H. Brunetto\inst{1}}


\address{Departamento de Informática -- Universidade Estadual de Maringá (UEM)\\
	Maringá -- PR -- Brasil
	%TODO: Inserir e-mail
	\email{ra94182@uem.br}
}

\begin{document}

	\maketitle

	\begin{abstract}
		This paper makes use of known trigonometric relations in order to develop a
		computational strategy that allows to calculate the sine of any angles of the
		trigonometric circle from the first six terms of the Taylor Series that defines it,
		so that	the result has ten decimal digits of precision in the minor possible computational
		time.
	\end{abstract}

	\begin{resumo}
		Este artigo faz uso de relações trigonométricas conhecidas a fim de desenvolver
		uma estratégia computacional que permita calcular o seno de quaisquer ângulos do
		círculo trigonométrico a partir dos seis primeiros termos da série de Taylor que
		a define de forma que o resultado disponha de dez dígitos decimais de precisão no
		menor tempo	computacional possível.
	\end{resumo}

\end{document}
