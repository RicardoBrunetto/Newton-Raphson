% Pacotes
\input{Preambulo-sbc.tex}

\sloppy

\title{Cálculo da Raiz Quadrada pelo Método de Newton-Raphson\\através do Padrão IEEE-745 para Ponto-Flutuante}

\author{Ricardo Henrique Brunetto\inst{1}}


\address{Departamento de Informática -- Universidade Estadual de Maringá (UEM)\\
	Maringá -- PR -- Brasil
	\email{ra94182@uem.br}
}

\begin{document}

	\maketitle

  \section{Introdução}
	Deseja-se obter uma maneira de estimar a raiz quadrada com base no método iterativo de Newton-Raphson, que
	estimativa de raízes de polinômios e equações transcedentais através da expansão da Série de McLaurin-Taylor.

	Tal método permite obter uma precisão específica através de convergência quadrática (se houver).

	Este artigo fará uma breve explicação do método de Newton-Raphson e do padrão IEEE 754 de armazenamento de ponto-flutuante
	que servirão de base para o desenvolvimento da estratégia utilizada. Não serão feitas explanações a respeito da Série de Taylor-McLaurin,
	visto que foge ao foco do artigo.

	\section{Fundamentação Teórica}
	\label{sec:fund_teor}
	Aqui serão abordados dois tópicos principais em que se baseará o cálculo da raiz quadrada posteriormente na Seção 3:
	o método de Newton-Raphson e o padrão IEEE 754.

	\subsection{Método de Newton-Raphson para determinação de raízes}
	O Método de Newton-Raphson se aplica aos polinômios e às equações transcedentais
	para determinar soluções de sistemas de equações simultâneas, com várias variáveis.
	No escopo deste artigo, seu uso será restringido para determinação de raízes em
	equações polinomiais com uma única variável e, portanto, será feito um breve resumo
	a respeito de seu funcionamento.

	O método de Newton-Raphson se principia com uma raiz aproximada, $x_k$, de uma equação
	$f(x) = 0$, e se vale da expansão em série de Taylor para formular um algoritmo computacional que define uma fórmula
	de recursão para uso.

	O desenvolvimento em série de Taylor de uma função nas vizinhanças do ponto $x_k$ tem a forma
	$$f(x) \approx f(x_k) + f'(x_k)(x - x_k)$$

	onde $f'(X_i)$ indica a derivada primeira de $f(x_k)$ calculada no ponto $(x_k, f(x_k))$.

	Dessa forma, dispõe-se da reta tangente ao gráfico da função $f(x)$, traçada no ponto $x_k$, para deteminação da próxima
	estimativa. Nota-se, que, ao admitir que a função $f(x)$ aproxima-se de sua(s) raiz(es), ocorre $f(x) = 0$ e, portanto, da equação acima:
	$$f(x_k) + f'(x_k) (x-x_k) = 0$$

	Alternativamente, pode-se expressar tal equação como sendo:
	$$f'(x_k) = \frac{f(x_k)}{x_k - x}$$

	Admitindo $x$ como a próxima estimativa, sendo esta mais satisfatória que $x_k$, tem-se a seguinte equação de recorrência, que caracteriza o Método de Newton-Raphson:
	$$x_{k+1} = x_k - \frac{f(x_k)}{f'(x_k)}\;\;\;\;\;\;(1)$$

	\subsection{Padrão IEEE 754}

	Aqui será realizada uma curta abordagem ao padrão IEEE 754 para armazenamento e aritmética binária. Não serão realizadas considerações
	a respeito da quantidade de bits nos diferentes tipos de precisão que o padrão oferece. Nesse caso, o foco torna-se especificamente a
	maneira como os números são acomodados na máquina para entendimento de como se pode manipular esta estrutura a fim de obter cálculos mais rápidos e precisos.

	Qualquer número $x$ pode ser representado por $x = mb^t$, onde $m$ é a mantissa, $b$ é a base e $t$ é o expoente.

	Um número decimal $x$ está representado no padrão IEEE 754 da seguinte forma:
	$$x_{10} = (-1)^{S_2}(1 + M_{10})2^{E_{10} - 1023}$$

	onde:
	\begin{itemize}
		\item $R_b$ implica que $R$ está na base $b$;
		\item $S$ é o valor do bit do sinal;
		\item $M$ é o valor da soma dos bits da mantissa;
		\item $E$ é o valor do expoente (onde $E\in\mathbb{N}$, pois é enviasado);
		\item $1023$ refere-se ao \textbf{BIAS} supondo 11 bits de precisão do expoente.
 	\end{itemize}

	A estrutura permite trabalhar com mantissa $M_2$ (supondo $m$ bits de precisão) de duas formas: inteira ($v_z(M) \in \mathbb{N}$) ou fracionada ($v_f(M) \in \mathbb{R}$).
	$$v_z(M) = \sum_{k=0}^{m-1}{M_b2^{b}}$$
	$$v_f(M) = \sum_{k=1}^{m}{M_k2^{-k}}$$

	onde $b$ e $k$ indexam o "vetor" de bits da mantissa conforme ilustra a figura abaixo.
	\begin{figure}[h]
		\centering
		\includegraphics{index_mant}
		\caption{Indexação da mantissa}
		\label{fig:index_mant}
	\end{figure}

	Salienta-se que, de qualquer forma, $v_z(M) = v_f(M)$.

	Assim, passam a existir duas maneiras de se trabalhar com a recuperação do valor armazenado:
	$$x = (-1)^{S}(1 + v_f(M))2^{E - 1023}$$
	ou
	$$x = (-1)^{S}(1 + \frac{v_z(M)}{2^m})2^{E - 1023}$$

	\section{Desenvolvimento}
	\label{sec:desenvolvimento}

	Deseja-se, nesta etapa, desenvolver um método para aplicar Newton-Raphson aproveitando-se do padrão IEEE 754 para calcular a raiz quadrada de um número $A$ em base $10$.

	Sabe-se que
	$$x = \sqrt{A} \implies x^2 = A\\\implies f(x) = x^2 - A = 0$$

	Nota-se que encontrar a raiz quadrada de $A$ significa encontrar o zero da função $f(x)$. Para tal utilizar-se-á Newton-Raphson.

	Da Seção 2.2, sabe-se que $A = m_a2^{e_a}$. Da mesma forma, $x = m_x2^{e_x}$.

	Ao aplicar Newton-Raphson em $f(x)$, tendo $x_k$ como estimativa inicial, tem-se:
	$$x_{k+1} = x_k - \frac{x_k^2 - A}{2x_k}$$
	que pode ser reescrito como:
	$$x_{k+1} = \frac{3x_k - x_k^2 + A}{2x_k}$$
	$$\equiv x_{k+1} = \frac{3x_k}{2x_k} - \frac{x_k^2}{2x_k} + \frac{A}{2x_k}$$
	$$\equiv x_{k+1} = \frac{3}{2} - \frac{x_k}{2} + \frac{A}{2x_k}$$
	$$\equiv x_{k+1} = 1.5 - \frac{x_k}{2} + \frac{A}{2x_k}$$

	Nota-se que $x_k = m_{x_k}2^{e_{x_k}}$ no padrão IEEE 754. Portanto,

	$$x_{k+1} = 1.5 - m_{x_k}2^{e_{x_k} - 1} + \frac{A}{m_{x_k}2^{e_{x_k} +1}}$$
	$$\equiv x_{k+1} = 1.5 - m_{x_k}2^{e_{x_k} - 1} + \frac{m_a2^{e_a}}{m_{x_k}2^{e_{x_k} +1}}$$
	$$\equiv x_{k+1} = 1.5 - m_{x_k}2^{e_{x_k} - 1} + \frac{m_a}{m_{x_k}}2^{e_a - e_{x_k} + 1}$$

	Adotando a extração fracionária do valor da mantissa, tem-se:
	$$x_{k+1} = 1.5 - m_{x_k}2^{e_{x_k} - 1} + \frac{m_a}{m_{x_k}}2^{e_a - e_{x_k} + 1}$$

	Nota-se que $\frac{m_a}{m_{x_k}}$ pode ser escrito como
	$$\frac{v_f(m_a)}{v_f(m_{x_k})}$$

	Substituindo $x_k$ por $x_i$:
	$$\frac{v_f(m_a)}{v_f(m_{x_i})} = \frac{\sum_{k=1}^{m}{(M_a)_k2^{-k}}}{\sum_{k=1}^{m}{({M_{x_i})_k2^{-k}}}}$$

	Note que a soma e subtração nos expoentes podem ser feitas com \textit{bit shift}.

	\section{Algoritmo}

	Antes de aplicar Newton-Raphson, definem-se algumas operações que podem ser feitas através do
	padrão IEEE-754. É importante especificar uma devida precisão. Existem duas formas básicas de fazer isso no método:
	especificar o número de iterações; especificar o erro relativo a ser tolerado (diferença nos resultados entre as iterações).
	Por motivos de precisão, escolher-se-á a segunda opção. Assim, define-se:
	\begin{verbatim}
		#define EPSILON 1e-20
	\end{verbatim}

	Em seguida, iniciar-se-á pela definição da função \verb*|doubleToIeee|,
	converte um valor real de ponto-flutuante na estrutura IEEE-754 definida
	previamente.

	\begin{verbatim}
		void doubleToIeee(double valor, int *s, int *exp, double *mant){
		  *exp = floor(log2(valor));
		  if(valor < 0){
		    valor = valor*(-1);
		    *s = 1;
		  }else{
		    *s = 0;
		  }
		  *mant = valor/(pow(2, *exp)) - 1.0;
		}
	\end{verbatim}

	Deseja-se, neste momento, deseja-se construir uma função que faça o cálculo da raiz
	quadrada. Para tanto, fazer-se-á uso da técnica de Newton-Raphson, discorrida na Seção
	\ref{sec:fund_teor}.

	Encontrar a raiz quadrada de um número $A$ significa encontrar a raiz da função $f(x) = x^2 - A$.
	Nota-se que $f'(x) = 2x$. Deseja-se, também, obter grande precisão. Para tanto, deve-se utilizar
	um valor reduzido de argumento. Nesse ínterim, uma boa estimativa inicial para o método de Newton-Raphson
	é a mantissa do valor no padrão IEEE-754, uma vez que a mantissa é sempre maior que zero e menor que $2$.

	Além do chute inicial, o padrão IEEE-754 pode ser utilizado no cálculo sumário do método Newton-Raphson. Isso
	significa que, em vez de calcular a raiz da função (valor decimal), calcula-se a mantissa da raiz da função.
	A manipulação do expoente fica a cargo da seguinte lógica: basta dividí-lo por $2$ (raiz quadrada), se for par;
	caso contrário, incrementa-se o expoente e corrige-se a mantissa (divide-a por $2$) e a normaliza, para então
	poder dividí-lo.

	Dessa forma, tem-se, seguindo o método de Newton-Raphson, o cálculo da Raiz Quadrada:
	\begin{verbatim}
		double raiz_quadrada (double mantissa, int expoente){
		  if(expoente & 1){
		    mantissa /= 2.0;
		    expoente++;
		  }

		  expoente /= 2;
		  double xk_1 = mantissa/2.0 + 1.0;
		  mantissa += 1.0;
		  double xk = 0, aux;

		  while(fabs(xk_1 - xk) > EPSILON){
		    xk = xk_1;
		    xk_1 = xk + (mantissa - xk*xk)/(2*xk);
		  }
		  return xk_1;
		}
	\end{verbatim}

	Dessa forma, calcula-se a raiz quadrada através do método Newton-Raphson com o padrão IEEE-754
	com precisão de 19 casas decimais.

	\section{Casos Teste}
	Testar-se-á a convergência para o cálculo das seguintes raízes:
	$$A = 25$$\\
	\begin{verbatim}
		#1 [xk]: 1.28125000000000000000	[xk+1]: 1.25038109756097570724
		#2 [xk]: 1.25038109756097570724	[xk+1]: 1.25000005807643410627
		#3 [xk]: 1.25000005807643410627	[xk+1]: 1.25000000000000133227
		#4 [xk]: 1.25000000000000133227	[xk+1]: 1.25000000000000000000
		#5 [xk]: 1.25000000000000000000	[xk+1]: 1.25000000000000000000

		Expoente: 2
		RAIZ CONVERGIDA: 1.25000000000000000000
	\end{verbatim}
	$$A = 81$$\\
	\begin{verbatim}
		#1 [xk]: 1.13281250000000000000	[xk+1]: 1.12502693965517241992
		#2 [xk]: 1.12502693965517241992	[xk+1]: 1.12500000032254554583
		#3 [xk]: 1.12500000032254554583	[xk+1]: 1.12500000000000000000
		#4 [xk]: 1.12500000000000000000	[xk+1]: 1.12500000000000000000

		Expoente: 3
		RAIZ CONVERGIDA: 1.12500000000000000000
	\end{verbatim}
	$$A = 5$$\\
	\begin{verbatim}
		#1 [xk]: 1.12500000000000000000	[xk+1]: 1.11805555555555558023
		#2 [xk]: 1.11805555555555558023	[xk+1]: 1.11803398895790206957
		#3 [xk]: 1.11803398895790206957	[xk+1]: 1.11803398874989490253
		#4 [xk]: 1.11803398874989490253	[xk+1]: 1.11803398874989490253

		Expoente: 1
		RAIZ CONVERGIDA: 1.11803398874989490253
	\end{verbatim}

	A precisão se mantém mesmo para grandes valores de $A$. Isso ocorre porque o cálculo da raiz
	é executado a partir da mantissa do valor de $A$ no padrão IEEE-754 e encontra a mantissa da raiz.

	\section{Conclusão}
	Em relação ao algoritmo desenvolvido, fazer-se-á uma análise a partir de cada iteração do método.
	Nota-se que a quantidade de iterações depende do tempo de convergência em relação à precisão
	desejada.

	Dessa forma, conclui-se que é possível aplicar Newton-Raphson onde cada iteração necessita de:
	\begin{itemize}
		\item 1 comparação
		\item 1 adição
		\item 2 subtrações
		\item 2 multiplicações
		\item 1 divisão
	\end{itemize}

	As operações de pré-processamento não são contabilizadas nesta análise, pois ocorrem somente uma
	única vez.

	Provavelmente é possível executar melhoras nesta técnica, talvez através do tratamento da mantissa
	como um número inteiro, para que se possa executar \textit{bit-shift}. Isso substituiria as multiplicações e
	divisões por $2$. Contudo, a conversão requer aplicação de uma fórmula que pode não compensar o uso do deslocamento de bits.
	Para tanto, é necessário um estudo mais profundo

\end{document}
